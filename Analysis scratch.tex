\documentclass[letterpaper,12pt,titlepage,oneside,final]{book}
\usepackage{graphicx}
\usepackage{setspace}
\usepackage{fullpage}
\usepackage{amsmath,amsthm,amssymb}

%%Packages for graphs
\usepackage{pgf}
\usepackage{tikz}
\usetikzlibrary{arrows,automata}
\usepackage[latin1]{inputenc}

%Package for multi column
\usepackage{multicol}

%%Pure laziness.
\newcommand{\N}{\mathbb{N}}
\newcommand{\Z}{\mathbb{Z}}
\newcommand{\R}{\mathbb{R}}
\newcommand{\Q}{\mathbb{Q}}
\newcommand{\C}{\mathbb{C}}



% TITLE PAGE GUBBINS
%----------------------------------------------------------------------
\newcommand{\thetitle}{Exercises for MTH 640 II}%Change title here
\newcommand{\coursecode}{MTH640}%Change course code here
\newcommand{\thedate}{\today}%Change the due date here


\begin{document}
% Title Page
%----------------------------------------------------------------------
\input{titlepage}
%----------------------------------------------------------------------


% Start Assignment
%----------------------------------------------------------------------

Sps, $\Omega = \{1,2,3\}$ and $\mathcal{F} = \{\varnothing, \Omega, \{1\}, \{2\}, \{3\}, \{1,2\}, \{1,3\}, \{2,3\}\}$, ie the power set of $\Omega$. Sps further that $\mathcal{C} = \{\{1\}, \{2,3\}\}\} \subset \mathcal{F}$.
Now, 
\begin{align*}
\sigma(\mathcal{C}) = \{&\mathcal{C}, \\
						&\mathcal{C}^{c} =\{ \varnothing, \Omega, \{2\}, \{3\}, \{1,2\}, \{1,3\}  \}, \\
						&\mathcal{C} \cap \mathcal{C}^{c} = \varnothing, \\
						&\mathcal{C} \cup \mathcal{C}^{c} = \mathcal{F} \}
\end{align*}

I have failed to show $\mathcal{C} \subset \sigma(\mathcal{C}) $, this is NOT a  $\sigma$-algebra over $\mathcal{C} $ since a $\sigma$-algebra over $\mathcal{C} $ must contain $\mathcal{C}$. However, the following is a a $\sigma$-algebra over $\mathcal{C}$:


\begin{align*}
\sigma(\mathcal{C}) = \{&\overbrace{\{1\}, \{2,3\}}^{ = \mathcal{C} }, \\
						&\overbrace{\{1, 2,3\}}^{{\{1\} \cup \{2,3\}}  } = \Omega, \\
						& \Omega^{c} = \varnothing, \}
\end{align*}.

We notice that  $\mathcal{C}$ is a subset of the above collection of sets. Furthermore, we find that the above collection of sets is a $\sigma$-algebra by Definition 1.1.

\end{document}
