\documentclass[letterpaper,12pt,titlepage,oneside,final]{book}
\usepackage{graphicx}
\usepackage{setspace}
\usepackage{fullpage}
\usepackage{amsmath,amsthm,amssymb}

%%Packages for graphs
\usepackage{pgf}
\usepackage{tikz}
\usetikzlibrary{arrows,automata}
\usepackage[latin1]{inputenc}

%Package for multi column
\usepackage{multicol}

%%Pure laziness.
\newcommand{\N}{\mathbb{N}}
\newcommand{\Z}{\mathbb{Z}}
\newcommand{\R}{\mathbb{R}}
\newcommand{\Q}{\mathbb{Q}}
\newcommand{\C}{\mathbb{C}}
\newcommand{\cis}{\text{cis}}
\newcommand{\matr}[1]{\mathbf{#1}} % undergraduate algebra version


% TITLE PAGE GUBBINS
%----------------------------------------------------------------------
\newcommand{\thetitle}{Assignment 1}%Change title here
\newcommand{\coursecode}{MTH719}%Change course code here
\newcommand{\thedate}{\today}%Change the due date here


\begin{document}
% Title Page
%----------------------------------------------------------------------
\input{titlepage}
%----------------------------------------------------------------------


% Start Assignment
%----------------------------------------------------------------------


\section*{1.a}
\begin{align*}
\text{Given, }\quad & u y''(t) + v y'(t) + w y(t) = f(t) \quad 1 < t < n+1 \\
&\text{and step size }h =\frac{a-b}{n}\\
\end{align*}
\begin{align*}
\matr{A} &= \begin{bmatrix}
	q & r & & 0\\
	p & \ddots & \ddots &  \\
	& \ddots & \ddots &  r \\
	0 &  & p & q  \end{bmatrix}
\end{align*}

\begin{align*}
\text{Where } p &= -u- v \times \frac{h}{2} \quad \\
q &= 2\times u - w\times h^2\\
r &= -u + v \times \frac{h}{2}
\end{align*}

\begin{align*}
\vec{X} = \begin{bmatrix}
	y(2) \\
	\vdots\\
	y(n) \end{bmatrix} \quad
\vec{B} = \begin{bmatrix}
	-f(2)\times h^2 -p\times \alpha \\
	-f(3)\times h^2\\
	\vdots\\
	-f(n-1)\times h^2\\
	-f(n)\times h^2 -p\times \beta \end{bmatrix}
\end{align*}

\begin{align*}
\text{Where } \alpha &= y(1) \quad \\
\beta &= y(n+1)\\
\text{and, } f(j) \text{ are known values}
\end{align*}
\cleardoublepage

\section*{1.c}
\begin{align*}
&\text{If computed values are}\quad
\hat{X} = \matr{A}^{-1}\vec{B}\\
&\text{and actual values are}\quad
\vec{X}\\
&\text{then accuracy for computed values is }\quad \epsilon =||\vec{X} - \hat{X}||\\
&\text{Where the smaller }\epsilon \text{ is preferred}. 
\end{align*}


\section*{2.a}
We can convert the room diagram into a transition matrix, where
the (i, j)-entry of the matrix corresponds to the probability that one person moves from from the jth room to the ith room.\\
\begin{align*}
\matr{A} &= \begin{bmatrix}
0.3333 &   0.2500  &     0 &   0.2500 &        0   &      0 &        0    &     0  &       0 \\
0.3333  &   0.2500  &  0.3333     &    0  &   0.2000 &     0   &      0  &       0   &      0\\
0 &   0.2500 &   0.3333  &       0   &      0  &  0.2500  &       0   &      0  &    0\\
0.3333 &  0 & 0 & 0.2500 &  0.2000   &  0  &  0.3333   & 0   & 0\\
0 &   0.2500   &   0 & 0.2500 &   0.2000 &   0.2500  &   0  &   0.2500  &   0 \\
0  & 0 &   0.3333 &  0 &   0.2000  &  0.2500 &  0   &  0 &  0.3333\\
0 &  0  & 0 &   0.2500  &  0 &   0 &   0.3333  &   0.2500 &    0 \\
0 &  0 & 0  &  0 &  0.2000 & 0 &   0.3333 &   0.2500 &   0.3333\\
0 &  0  &  0  &  0 &        0  &  0.2500  &       0  &  0.2500 &   0.3333
 \end{bmatrix}
\end{align*}
Note that all of the columns add up to one. 
\\
\\
Let the state of our system be represented by a probability vector
\begin{align*}
\vec{x_{0}} =  \begin{bmatrix}
	x_{1} \\
	\vdots\\
	x_{9} \end{bmatrix}
\end{align*}
where each entry represents the probability of being in that room at $t_{0}$.
\\
\\ 
If each time interval is represented by 15 minutes, then at $t_1$
\begin{align*}
\vec{x_{1}} &= A^1\times \vec{x_0}\\ =& A^1 \times 
\begin{bmatrix}
30/100 \\ 40/100 \\ 0 \\ 0 \\ 10/100 \\ 0 \\ 0 \\ 20/100 \\ 0  \end{bmatrix}
= \begin{bmatrix}
    0.2000\\
    0.2200\\
    0.1000\\
    0.1200\\
    0.1700\\
    0.0200\\
    0.0500\\
    0.0700\\
    0.0500\\
    \end{bmatrix}\\
\end{align*}
\\
\\
And the number of people in each room is
\begin{align*}
= 100\times \begin{bmatrix}
    0.2000\\
    0.2200\\
    0.1000\\
    0.1200\\
    0.1700\\
    0.0200\\
    0.0500\\
    0.0700\\
    0.0500\\
    \end{bmatrix} = 
    \begin{bmatrix}
   20\\
   22\\
   10\\
   12\\
   17\\
    2\\
    5\\
    7\\
    5\\
    \end{bmatrix}
\end{align*}

\section*{2.b}
We know that $x_n = A x_{n-1} = \hdots = A^n x_0 $ . Given $x_j$, we need to find $x_{j-1}$.
\begin{align*}
x_j &=  A \times x_{j-1} &x_j = \begin{bmatrix}    
	0.1295\\
    0.0518\\
    0.1554\\
    0.0777\\
    0.2850\\
    0.1036\\
    0.1295\\
    0.0518\\
    0.0155\\
    \end{bmatrix}\\
\end{align*}
So, the probability distribution of the rooms at $t_{j-1}$ is $x_{j-1} =  A^{-1} \times x_{j}$ and number of people in the each room at $t_{j-1}$ is $1100 \times x_{j-1}$ 
\begin{align*}
1100\times x_{j-1} = 1100\times A^{-1}x_j =
\begin{bmatrix} 
   125\\
    50\\
   150\\
    75\\
   275\\
   100\\
   125\\
    50\\
   150\\ 
\end{bmatrix}
\end{align*}

\section*{2.c}
We know that $x_n = A^n x_0 $. \\
We set $
x_{0} = \begin{bmatrix}
   1\\
    0\\
   0\\
    0\\
   0\\
   0\\
   0\\
    0\\
   0\\ 
\end{bmatrix}$. 
Then at $t_4$ is the first time that room 9 is filled. 
\\
\\
$x_4 = A^4 \times x_0 = \begin{bmatrix}
    0.1644\\
    0.1654\\
    0.0913\\
    0.1654\\
    0.1578\\
    0.0669\\
    0.0913\\
    0.0669\\
    0.0306 \\
\end{bmatrix}$
. Where each row is the distribution of the population at $t_4$.

\cleardoublepage
\section*{2.d}
We see that 
\begin{align*}
&\lim_{n \to \infty} A^n = \\
\\
&\begin{bmatrix}
0.0909 & 0.0909 & 0.0909 & 0.0909 & 0.0909 & 0.0909 & 0.0909 & 0.0909 & 0.0909\\
0.1212 & 0.1212 & 0.1212 & 0.1212 & 0.1212 & 0.1212 & 0.1212 & 0.1212 & 0.1212\\
0.0909 & 0.0909 & 0.0909 & 0.0909 & 0.0909 & 0.0909 & 0.0909 & 0.0909 & 0.0909\\
0.1212 & 0.1212 & 0.1212 & 0.1212 & 0.1212 & 0.1212 & 0.1212 & 0.1212 & 0.1212\\
0.1515 & 0.1515 & 0.1515 & 0.1515 & 0.1515 & 0.1515 & 0.1515 & 0.1515 & 0.1515\\
0.1212 & 0.1212 & 0.1212 & 0.1212 & 0.1212 & 0.1212 & 0.1212 & 0.1212 & 0.1212\\
0.0909 & 0.0909 & 0.0909 & 0.0909 & 0.0909 & 0.0909 & 0.0909 & 0.0909 & 0.0909\\
0.1212 & 0.1212 & 0.1212 & 0.1212 & 0.1212 & 0.1212 & 0.1212 & 0.1212 & 0.1212\\
0.0909 & 0.0909 & 0.0909 & 0.0909 & 0.0909 & 0.0909 & 0.0909 & 0.0909 & 0.0909\\
\end{bmatrix}
\end{align*}
\\
\\
Hence the Markov chain is convergent, ie regardless of the initial vector, all nonzero vectors will converge to $\lim_{n \to \infty} x_{n} = \lim_{n \to \infty} A^n \times x_0 $\\
\\
$\lim_{x \to \infty} x_{n} = 
\begin{bmatrix}    
    0.0909\\
    0.1212\\
    0.0909\\
    0.1212\\
    0.1515\\
    0.1212\\
    0.0909\\
    0.1212\\
    0.0909\\
\end{bmatrix}$. Thus it is also the distribution of resources for each room.
\end{document}
