\documentclass[letterpaper,12pt,titlepage,oneside,final]{book}
\usepackage{graphicx}
\usepackage{setspace}
\usepackage{fullpage}
\usepackage{amsmath,amsthm,amssymb}

%%Packages for graphs
\usepackage{pgf}
\usepackage{tikz}
\usetikzlibrary{arrows,automata}
\usepackage[latin1]{inputenc}

%Package for multi column
\usepackage{multicol}

%%Pure laziness.
\newcommand{\N}{\mathbb{N}}
\newcommand{\Z}{\mathbb{Z}}
\newcommand{\R}{\mathbb{R}}
\newcommand{\Q}{\mathbb{Q}}
\newcommand{\C}{\mathbb{C}}
\newcommand{\cis}{\text{cis}}
\newcommand{\matr}[1]{\mathbf{#1}} % undergraduate algebra version


% TITLE PAGE GUBBINS
%----------------------------------------------------------------------
\newcommand{\thetitle}{Assignment 2}%Change title here
\newcommand{\coursecode}{MTH719}%Change course code here
\newcommand{\thedate}{\today}%Change the due date here


\begin{document}
% Title Page
%----------------------------------------------------------------------
\input{titlepage}
%----------------------------------------------------------------------


% Start Assignment
%----------------------------------------------------------------------


\section*{1.a}
To show \quad $\frac{A + A^T}{2}$\quad is symmetric, we need to show: \\
$\frac{A + A^T}{2} = \left(\frac{A + A^T}{2} \right)^T$ \\
\begin{align*}
\text{Now, }\quad \left(\frac{A + A^T}{2} \right)^T 
&= \frac{A^T + (A^T)^T}{2} 
= \frac{A^T + A}{2}
= \frac{A + A^T}{2}\\
\text{Hence,}\quad
\frac{A + A^T}{2} &= \left(\frac{A + A^T}{2} \right)^T
\end{align*}

\begin{align*}
\text{Now, we want to show: }\\
X^T\left(\frac{A + A^T}{2}\right)X &= X^TAX\\
\text{We know, }\quad \frac{A + A^T}{2} &= \left(\frac{A + A^T}{2} \right)^T\\
\text{Then, }\quad 
X^T\left(\frac{A + A^T}{2}\right)X&= X^T\left(\frac{A + A^T}{2} \right)^T X\quad \text{by left and right matrix multiplication}\\
&=\left[\left(\frac{A + A^T}{2} \right)X\right]^T X \quad\text{by properties of transpose}\\
&= \left(\frac{AX + A^TX}{2} \right)^T X \quad\text{by distributive property}\\
&= \left(\frac{X^T A^T + X^T A}{2}\right)X\quad\text{by properties of transpose}\\
&=\frac{X^T A^T X + X^T A X}{2}\quad\text{by distributive property}\\
\text{Now, if }\quad A = A^T\quad\text{then: }\\
\frac{X^T A^T X + X^T A X}{2}
&=\frac{X^T A^T X + X^T A^T X}{2}\\
& = \frac{2X^T A^T X}{2} = X^T A^T X 
\end{align*}

\cleardoublepage

\section*{1.b}
\text{Suppose A is a positive definite matrix, then:}
\begin{align*}
X^T A X &= X^TLDUX = X^TLDL^TX = X^TU^TDUX\\
&\text{by definition of positive definite LDU factorization}&\\
& =(UX)^T D (UX)\\
\end{align*}
So, if the diagonal elements of D are positive numbers $X^T A X$ is positive definite $\forall X$, since $X^T A X$ also has a positive definite LDU factorization.


\section*{1.c}
Using Taylor's theorem in $\R ^n$ we know that $f(x) = f(x_0) + (x-x_0)^Tg(x_0) + (x-x_0)^T$



\end{document}
